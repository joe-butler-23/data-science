%!TEX root = CS818_assessment.tex
Obesity is a critical public health challenge, with links to elevated risk of type 2 diabetes, cardiovascular disease, and certain cancers \cite{Kinlen2018}. Epidemiological research has long established that lifestyle factors - most notably diet and physical activity — are closely associated with body mass index (BMI) and general metabolic health \cite{Bergens2020}. These factors subsequently form the basis of much public health advice on obesity. However, the strength and nature of these associations can be heterogeneous across different populations. For example physically active and metabolically healthy individuals may nevertheless be obese \cite{Alcazar2021}, and this combination is more prevalent amongst females and younger age groups \cite{Bluher2020}. Similarly, whilst a diet rich in saturated fats has been linked to increased obesity risk \cite{Wang2020}, countries like France maintain comparatively low levels of obesity with comparatively high saturated fat intake \cite{Ducrot2018}.

This heterogeneity presents a challenge for public health officials seeking to identify the most impactful interventions. It also challenges conventional statistical methods, which often assume homogenous relationships between variables \cite{Hoekstra2012}. More advanced analytical techniques are required to disentangle these complex relationships between lifestyle factors and obesity.

This study adds to the debate by exploring the multifactorial nature of obesity with both supervised and unsupervised learning techniques. The findings highlight key lifestyle factors associated with obesity as well as their interrelationships, which may permit more targeted public health interventions.
