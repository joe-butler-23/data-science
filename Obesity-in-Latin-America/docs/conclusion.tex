%!TEX root = CS818_assessment.tex
This analysis reconfirms that obesity risk is not monolithic, but rather varies greatly across different subpopulations defined by both lifestyle and demographic factors. Whilst traditional metrics remain critical, a richer understanding of obesity and its multifactorial nature is achieved when more sophisticated statistical tools are used. The findings of the study also suggest some potential areas of focus for public health interventions, notably physical activity. Use of public transport and high screentime were two of the study's most significant variables relating to obesity risk, with these factors even potentially nullifying some of the benefits of other healthy behaviours. One potential intervention could therefore be promoting active transport. The initial exploratory analysis showed respondents overwhelmingly rely on public transport, and to a lesser degree automobiles. The low prevalence of cycling and walking at present suggests this could be an impactful way to reduce overall sedentary time in these countries. 

Despite these strengths, there are also a number of weaknesses and limitations in the research. The analysis is based on a rich dataset which is nevertheless made up primarily of synthetic data, and the utility and generalisability of the data is dependent on the strength of the data generation methods used. Finally, this study is cross-sectional, examining data at a single point in time. The variables studied here and their relationships may vary over time, which could impact the conclusions drawn. Whilst particular relationships, such as that between high vegetable consumption and the highest BMI, have already been highlighted as potentially useful avenues for future research, further studies should also include longitudinal studies that observe how these factors interact over extended time-frames. Similarly, rerunning the analysis on other datasets could further validate the findings, improve the generalisability of results, and address some of the data quality concerns outlined above. In this way, we can continue to build on the analysis set out here and shed further light on a complex and multifaceted public health issue. 
